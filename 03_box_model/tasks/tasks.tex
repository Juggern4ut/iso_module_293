\documentclass[a4paper,12pt]{article}
\usepackage[utf8]{inputenc}
\usepackage[ngerman]{babel}
\usepackage{amsmath}
\usepackage{parskip}

\setlength{\parindent}{0pt} % Keine Einrückungen

\title{Modul 293 - Aufgabenblatt 03}

\author{von Lukas Meier}

\date{Unterricht vom 09.12.2025}

\begin{document}

\maketitle

\section{Aufgaben}

In dieser Lektion vertiefen Sie Ihr Wissen über das CSS Box-Modell sowie das Layout-Verhalten von Inline- und Block-Elementen. Arbeiten Sie die folgenden Aufgaben der Reihe nach durch und beobachten Sie die Auswirkungen Ihrer Änderungen jeweils im Browser.

\subsection{Vorbereitung}

Öffnen Sie Ihr bestehendes Projekt aus den vorherigen Lektionen (index.html, contact.html sowie styles.css). Stellen Sie sicher, dass Ihre CSS-Datei korrekt im \texttt{head}-Bereich der HTML-Dateien eingebunden ist.

\subsection{Ein einfaches Box-Element}

Erstellen Sie in der Datei index.html unterhalb des Headers ein neues \texttt{div}-Element. Vergeben Sie diesem Element die Klasse \texttt{box}. Befüllen Sie das \texttt{div} mit einem kurzen Text Ihrer Wahl.

Erstellen Sie in der Datei styles.css eine CSS-Regel für die Klasse \texttt{box} und definieren Sie folgende Eigenschaften:

\begin{itemize}
    \item Breite: 300px
    \item Hintergrundfarbe: hellgrau
\end{itemize}

Laden Sie die Seite im Browser neu. Wie verhält sich das Element im Layout?

\subsection{Padding und Border}

Erweitern Sie die bestehende CSS-Regel der Klasse \texttt{box} um folgende Eigenschaften:

\begin{itemize}
    \item Padding: 20px
    \item Border: 2px solid schwarz
\end{itemize}

Beobachten Sie, wie sich der Abstand zwischen Text und Rahmen verändert. Welche Rolle spielt das Padding?

\subsection{Margin}

Ergänzen Sie der \texttt{box}-Klasse nun noch eine Margin von 30px.

Beobachten Sie den Abstand des Elements zu anderen Elementen auf der Seite. Beschreiben Sie den Unterschied zwischen Padding und Margin.

\subsection{Gesamtgrösse der Box}

Messen oder schätzen Sie die effektive Breite der Box im Browser. Ist diese grösser als 300px? Begründen Sie Ihre Beobachtung.

\subsection{box-sizing}

Ergänzen Sie in Ihrer CSS-Datei folgende Regel ganz oben:

\begin{verbatim}
* {
  box-sizing: border-box;
}
\end{verbatim}

Laden Sie die Seite neu. Was ändert sich an der Gesamtgrösse der Box? Erklären Sie den Unterschied zum vorherigen Verhalten.

\subsection{Block-Elemente untersuchen}

Erstellen Sie zwei weitere \texttt{div}-Elemente unterhalb der bestehenden Box, ebenfalls mit der Klasse \texttt{box}.

Beobachten Sie:
\begin{itemize}
    \item Wie werden die Elemente untereinander angeordnet?
    \item Welche Breite nehmen die Elemente ein?
\end{itemize}

Warum ist dieses Verhalten typisch für Block-Elemente?

\subsection{Inline-Elemente}

Erstellen Sie innerhalb eines Paragrafen ein \texttt{span}-Element und geben Sie diesem die Klasse \texttt{inline-test}. Schreiben Sie etwas Text in das \texttt{span}.

Definieren Sie in der CSS-Datei folgende Eigenschaften für \texttt{.inline-test}:

\begin{itemize}
    \item Hintergrundfarbe: gelb
    \item Breite: 200px
    \item Margin-top: 30px
\end{itemize}

Welche dieser Eigenschaften haben einen sichtbaren Effekt? Welche nicht? Begründen Sie Ihre Beobachtung.

\subsection{Inline zu Block ändern}

Ändern Sie nun den Display-Typ des \texttt{span}-Elements auf \texttt{block}.

\begin{verbatim}
display: block;
\end{verbatim}

Was ändert sich am Verhalten des Elements? Welche CSS-Eigenschaften funktionieren nun wie erwartet?

\subsection{inline-block}

Setzen Sie den Display-Typ des \texttt{span}-Elements nun auf \texttt{inline-block}. Ergänzen Sie zusätzlich ein Padding von 10px.

Vergleichen Sie das Verhalten mit \texttt{inline} und \texttt{block}. In welchen Situationen könnte \texttt{inline-block} sinnvoll sein?

\subsection{Navigation untersuchen}

Untersuchen Sie die Links in Ihrer Navigation (\texttt{a}-Tags). Handelt es sich dabei standardmässig um Inline- oder Block-Elemente?

Ändern Sie den Display-Typ der Links auf \texttt{block}. Beobachten Sie den Effekt auf:
\begin{itemize}
    \item Klickfläche
    \item Anordnung
\end{itemize}

\subsection{Layout-Debugging}

Öffnen Sie die zur Verfügung gestellte HTML-Datei im Browser. Die Seite enthält mehrere absichtlich eingebaute Layout-Fehler. Ziel dieser Aufgabe ist es, diese Fehler zu erkennen, zu verstehen und zu beheben.

\subsubsection{Navigation analysieren}

Betrachten Sie die Navigation im oberen Bereich der Seite.

\begin{itemize}
    \item Wie verhalten sich die Links im Layout?
    \item Wie gross ist die Klickfläche der einzelnen Links?
\end{itemize}

Warum ist dieses Verhalten typisch für den aktuellen Display-Typ der \texttt{a}-Tags?

\subsubsection{Navigation verbessern}

Passen Sie das CSS so an, dass:
\begin{itemize}
    \item jeder Link in einer eigenen Zeile dargestellt wird oder
    \item die Klickfläche klar erkennbar und gross ist
\end{itemize}

Welcher Display-Typ eignet sich hierfür besonders gut?

\subsubsection{Box-Modell untersuchen}

Betrachten Sie die beiden Boxen im Hauptbereich der Seite.

\begin{itemize}
    \item Welche Breite haben die Boxen effektiv im Browser?
    \item Entspricht diese Breite dem definierten Wert von 300px?
\end{itemize}

Begründen Sie Ihre Beobachtung anhand des CSS Box-Modells.

\subsubsection{Box-Modell korrigieren}

Passen Sie das CSS so an, dass die Gesamtbreite der Boxen tatsächlich 300px beträgt, inklusive Padding und Border.

Welche CSS-Eigenschaft wird dafür benötigt?

\subsubsection{Inline-Element analysieren}

Betrachten Sie das hervorgehobene Wort im Paragrafen.

\begin{itemize}
    \item Welche CSS-Eigenschaften haben keinen sichtbaren Effekt?
    \item Warum ist das bei diesem Element so?
\end{itemize}

Beziehen Sie sich dabei auf den Display-Typ des Elements.

\subsubsection{Inline-Element korrigieren}

Ändern Sie den Display-Typ des \texttt{span}-Elements so, dass:
\begin{itemize}
    \item Breite und Höhe angewendet werden
    \item das Element weiterhin im Textfluss bleibt
\end{itemize}

Welcher Display-Wert eignet sich hierfür?

\subsubsection{Reflexion}

Beschreiben Sie in 2--3 Sätzen:
\begin{itemize}
    \item einen Fehler, den Sie gefunden haben
    \item wie Sie diesen behoben haben
    \item welche Regel oder welches Konzept dahinter steckt
\end{itemize}

\subsection{Bonus: Layout Experiment}

Experimentieren Sie mit folgenden CSS-Eigenschaften:

\begin{itemize}
    \item Unterschiedliche Margin- und Padding-Werte
    \item Display-Typen bei verschiedenen Elementen
    \item Kombination von \texttt{width}, \texttt{height} und \texttt{display}
\end{itemize}

Notieren Sie sich mindestens zwei Beobachtungen, die Sie über das Layout-Verhalten von HTML-Elementen gemacht haben.

\end{document}