\documentclass[a4paper,12pt]{article}
\usepackage[utf8]{inputenc}
\usepackage[ngerman]{babel}
\usepackage{amsmath}
\usepackage{parskip}

\setlength{\parindent}{0pt}

\title{Modul 293 - Aufgabenblatt 08}
\author{von Lukas Meier}
\date{Unterricht vom 10.02.2026}

\begin{document}
\section*{Aufgabenblatt: Repetition HTML und CSS}

\subsection*{Teil 1: Grundlagen-Check}

Beantworte die folgenden Fragen schriftlich:

\begin{itemize}
  \item Was ist der Zweck von \texttt{<!DOCTYPE html>}?
  \item Was ist der Unterschied zwischen \texttt{id} und \texttt{class}?
  \item Welche Aufgabe hat das \texttt{head}-Element?
\end{itemize}

\subsection*{Teil 2: HTML-Struktur verbessern}

Der folgende Code ist unsauber:

\begin{verbatim}
<div>
  <div>Meine Seite</div>
  <div>
    <div><a href="#">Home</a></div>
    <div><a href="#">Kontakt</a></div>
  </div>
</div>
\end{verbatim}

\begin{itemize}
  \item Überführe den Code in eine semantisch sinnvolle Struktur.
  \item Verwende dabei mindestens: \texttt{header}, \texttt{nav}, \texttt{main}, \texttt{footer}.
\end{itemize}

\subsection*{Teil 3: CSS und Box Model}

Gegeben ist folgendes CSS:

\begin{verbatim}
.card {
  width: 200px;
  padding: 20px;
  border: 4px solid black;
  margin: 10px;
}
\end{verbatim}

\begin{itemize}
  \item Welche Bereiche gehören zum Box Model?
  \item Wie verändert sich die sichtbare Grösse der Box durch Padding und Border?
\end{itemize}

Erstelle danach eine eigene Box-Klasse mit anderen Werten.

\subsection*{Teil 4: Listen, Navigation und Links}

Erstelle eine Navigation mit einer ungeordneten Liste und mindestens vier Links.

\begin{itemize}
  \item Einer der Links soll auf eine externe Seite zeigen.
  \item Verwende für den externen Link \texttt{target="\_blank"}.
  \item Formuliere klare, beschreibende Linktexte.
\end{itemize}

\subsection*{Teil 5: Positionierung}

Erkläre den Unterschied zwischen:

\begin{itemize}
  \item \texttt{position: relative;}
  \item \texttt{position: absolute;}
  \item \texttt{position: fixed;}
\end{itemize}

Erstelle ein Beispiel mit einem relativ positionierten Container und einem absolut positionierten Element darin.

\subsection*{Teil 6: Flexbox anwenden}

Baue einen Bereich mit drei Karten nebeneinander.

Vorgaben:
\begin{itemize}
  \item Nutze \texttt{display: flex;} im Container
  \item Verwende \texttt{gap} für Abstände
  \item Zentriere die Karten vertikal oder horizontal mit den passenden Flex-Eigenschaften
\end{itemize}

\subsection*{Teil 7: Gesamtaufgabe}

Erstelle eine kleine Webseite mit:

\begin{itemize}
  \item Header mit Titel
  \item Navigation mit Liste und Links
  \item Hauptbereich mit zwei Content-Boxen
  \item Footer mit kurzem Text
\end{itemize}

Anforderungen:
\begin{itemize}
  \item Semantische HTML-Struktur
  \item Sauberes CSS mit Klassen
  \item Sichtbarer Einsatz von Box Model, Positionierung \textbf{oder} Flexbox
\end{itemize}

\subsection*{Teil 8: Reflexion}

Beantworte die folgenden Fragen:

\begin{itemize}
  \item Welches Thema aus den bisherigen Lektionen sitzt bereits sicher?
  \item Wo brauchst du noch Übung?
  \item Welche zwei konkreten Lernziele setzt du dir für die nächste Lektion?
\end{itemize}

\end{document}
