\documentclass[a4paper,12pt]{article}
\usepackage[utf8]{inputenc}
\usepackage[ngerman]{babel}
\usepackage{amsmath}
\usepackage{parskip}

\setlength{\parindent}{0pt}

\title{Modul 293 - Aufgabenblatt 08}
\author{von Lukas Meier}
\date{Unterricht vom 10.02.2026}

\begin{document}

\maketitle

\section*{Aufgabenblatt: Repetition HTML und CSS}

Bearbeiten Sie die Aufgaben in der vorgegebenen Reihenfolge. Die Aufgaben werden pro Thema schrittweise anspruchsvoller.

\section*{Thema 1: HTML Grundlagen}

\subsection*{Aufgabe 1.1 -- Begriffe sicher unterscheiden}
Erklären Sie in eigenen Worten:
\begin{itemize}
  \item Was ist ein HTML-Tag?
  \item Was ist ein Attribut?
  \item Was ist der Unterschied zwischen Start-Tag, End-Tag und Inhalt?
\end{itemize}

Geben Sie pro Begriff ein kurzes Beispiel an.

\subsection*{Aufgabe 1.2 -- Tags erkennen und einordnen}
Analysieren Sie den folgenden Code:

\begin{verbatim}
<a href="https://example.com" target="_blank">Zur Website</a>
<img src="bild.jpg" alt="Profilbild">
<section class="intro">Willkommen</section>
\end{verbatim}

Bearbeiten Sie dazu:
\begin{itemize}
  \item Markieren Sie alle Tags.
  \item Markieren Sie alle Attribute.
  \item Notieren Sie, welches Element selbstschliessend ist.
\end{itemize}

\subsection*{Aufgabe 1.3 -- Selbstschliessende Tags}
Erstellen Sie eine kurze HTML-Struktur (ohne CSS und nur den body) mit:
\begin{itemize}
  \item einer Hauptüberschrift
  \item einem Absatz
  \item einem Zeilenumbruch
  \item einem Bild mit sinnvollem \texttt{alt}-Attribut
  \item einer horizontalen Linie
\end{itemize}

Verwenden Sie dabei mindestens drei selbstschliessende Tags.

\subsection*{Aufgabe 1.4 -- Semantik verbessern}
Gegeben ist folgender HTML-Ausschnitt:

\begin{verbatim}
<div>
  <div>Meine Lernseite</div>
  <div>
    <div>Start</div>
    <div>Kontakt</div>
  </div>
  <div>
    <div>Heute üben wir HTML.</div>
  </div>
</div>
\end{verbatim}

Ersetzen Sie die unspezifischen \texttt{div}-Elemente durch semantisch passende Tags (z.\,B. \texttt{header}, \texttt{nav}, \texttt{main}, \texttt{section}, \texttt{footer}).

\subsection*{Aufgabe 1.5 -- Fehleranalyse}
Im folgenden Code stecken mehrere konzeptionelle Fehler:

\begin{verbatim}
<section>
  <h1 class="titel">Wochenplan
  <p id="text" "wichtig">Montag: CSS wiederholen</p>
  <img source="plan.png">
</section>
\end{verbatim}

Bearbeiten Sie:
\begin{itemize}
  \item Finden Sie die vier Fehler.
  \item Korrigieren Sie den Code vollständig.
  \item Begründen Sie jede Korrektur in einem Satz.
\end{itemize}

\subsection*{Aufgabe 1.6 -- Transferaufgabe HTML}
Erstellen Sie eine vollständige Mini-Seite mit:
\begin{itemize}
  \item \texttt{header} mit Titel
  \item \texttt{nav} mit mindestens drei Links
  \item \texttt{main} mit zwei \texttt{section}-Bereichen
  \item mindestens einem selbstschliessenden Tag
  \item \texttt{footer} mit Copyright-Text
\end{itemize}

Achten Sie auf korrekte Verschachtelung und sinnvolle Semantik.

\section*{Thema 2: CSS Selektoren}

\subsection*{Aufgabe 2.1 -- Grundprinzipien}
Beantworten Sie schriftlich:
\begin{itemize}
  \item Wie spricht man ein Element in CSS nur über den Tag-Namen an?
  \item Wie spricht man eine Klasse an?
  \item Wie spricht man eine ID an?
\end{itemize}

Geben Sie je ein kurzes Beispiel (\texttt{HTML + CSS}) an.

\subsection*{Aufgabe 2.2 -- Selektoren zuordnen}
Gegeben ist dieser HTML-Code:

\begin{verbatim}
<h1 id="title">Modul 293</h1>
<p class="info">Einführung</p>
<p class="info">Repetition</p>
<button class="info" id="start-btn">Start</button>
\end{verbatim}

Notieren Sie für die folgenden Selektoren jeweils, welche Elemente getroffen werden:
\begin{itemize}
  \item \texttt{p}
  \item \texttt{.info}
  \item \texttt{\#title}
  \item \texttt{button.info}
\end{itemize}

\subsection*{Aufgabe 2.3 -- Mehrere Elemente in einem Selektor}
Erstellen Sie CSS-Regeln, sodass:
\begin{itemize}
  \item \texttt{h1}, \texttt{h2} und \texttt{h3} dieselbe Textfarbe erhalten
  \item alle \texttt{p}- und \texttt{li}-Elemente dieselbe Schriftgrösse erhalten
  \item die Klasse \texttt{note} und die ID \texttt{warning} dieselbe Hintergrundfarbe erhalten
\end{itemize}

Nutzen Sie dafür jeweils gruppierte Selektoren mit Komma.

\subsection*{Aufgabe 2.4 -- Kombinierte Selektoren}
Ergänzen Sie passende CSS-Regeln für folgende Anforderungen:
\begin{itemize}
  \item Nur \texttt{p}-Elemente mit Klasse \texttt{info} sollen blau sein.
  \item Nur das Element mit ID \texttt{menu} soll einen Rand erhalten.
  \item Nur \texttt{li}-Elemente innerhalb einer Klasse \texttt{nav} sollen fett sein.
\end{itemize}

Erklären Sie kurz, warum ein allgemeiner Selektor wie \texttt{.info} hier zu ungenau wäre.

\subsection*{Aufgabe 2.5 -- Fehler finden und korrigieren}
Korrigieren Sie den CSS-Code:

\begin{verbatim}
title {
  color: red;
}

#.intro {
  font-size: 20px;
}

p, .hinweis #wichtig {
  background: yellow;
}
\end{verbatim}

Bearbeiten Sie:
\begin{itemize}
  \item Korrigieren Sie alle Syntax- und Logikfehler.
  \item Erklären Sie pro Zeile kurz, was falsch war.
\end{itemize}

\subsection*{Aufgabe 2.6 -- Transferaufgabe Selektoren}
Erstellen Sie einen HTML-Ausschnitt mit mindestens:
\begin{itemize}
  \item 2 Überschriften
  \item 3 Absätzen
  \item 1 Liste mit 4 Einträgen
  \item 1 Button
\end{itemize}

Erstellen Sie danach CSS mit:
\begin{itemize}
  \item mindestens einem Tag-Selektor
  \item mindestens zwei Klassen-Selektoren
  \item mindestens einem ID-Selektor
  \item mindestens zwei gruppierten Selektoren
  \item mindestens einem kombinierten Selektor (z.\,B. \texttt{p.info})
\end{itemize}

\section*{Thema 3: CSS Box-Modell}

\subsection*{Aufgabe 3.1 -- Theorie verstehen}
Beschreiben Sie den Aufbau des Box-Modells in der korrekten Reihenfolge von innen nach aussen.

Erklären Sie anschliessend in eigenen Worten:
\begin{itemize}
  \item Was ist \texttt{padding}?
  \item Was ist \texttt{margin}?
  \item Welche CSS-Properties werden für diese verwendet?
\end{itemize}

\subsection*{Aufgabe 3.2 -- Box aufbauen}
Erstellen Sie eine Klasse \texttt{.card} mit:
\begin{itemize}
  \item \texttt{width: 280px}
  \item \texttt{padding: 16px}
  \item \texttt{border: 2px solid black}
  \item \texttt{margin: 20px}
\end{itemize}

Skizzieren Sie zusätzlich, welcher Bereich Inhalt, Padding, Border und Margin ist.

\subsection*{Aufgabe 3.3 -- Margin vs. Padding gezielt testen}
Erstellen Sie zwei gleich grosse Boxen nebeneinander oder untereinander.

Bearbeiten Sie schrittweise:
\begin{itemize}
  \item Erhöhen Sie nur das \texttt{padding} und beobachten Sie den Effekt.
  \item Setzen Sie das \texttt{padding} zurück und erhöhen Sie nur die \texttt{margin}.
  \item Beschreiben Sie den Unterschied in mindestens drei Sätzen.
\end{itemize}

\subsection*{Aufgabe 3.4 -- Block und Inline vergleichen}
Untersuchen Sie je ein \texttt{div}- und ein \texttt{span}-Element.

Bearbeiten Sie:
\begin{itemize}
  \item Welche Standard-\texttt{display}-Werte haben die beiden Elemente?
  \item Welche Eigenschaften (\texttt{width}, \texttt{height}, vertikale \texttt{margin}) wirken bei \texttt{span} standardmässig nicht wie erwartet?
  \item Ändern Sie das \texttt{span} auf \texttt{display: inline-block;} und vergleichen Sie erneut.
\end{itemize}

\subsection*{Aufgabe 3.5 -- Rechnen mit dem Box-Modell}
Gegeben:
\begin{itemize}
  \item \texttt{width: 300px}
  \item \texttt{padding-left: 20px}
  \item \texttt{padding-right: 20px}
  \item \texttt{border-left: 3px}
  \item \texttt{border-right: 3px}
\end{itemize}

Berechnen Sie die gesamte sichtbare Breite:
\begin{itemize}
  \item einmal mit \texttt{box-sizing: content-box}
  \item einmal mit \texttt{box-sizing: border-box}
\end{itemize}

Erklären Sie den Unterschied.

\subsection*{Aufgabe 3.6 -- Transferaufgabe Box-Modell}
Erstellen Sie ein kleines Layout mit drei Karten (\texttt{article}-Elemente):
\begin{itemize}
  \item Jede Karte enthält Titel und Text.
  \item Jede Karte hat sichtbare Unterschiede bei Margin und Padding.
  \item Mindestens eine Karte verwendet \texttt{display: inline-block}.
  \item Für alle Elemente soll \texttt{box-sizing: border-box;} gesetzt sein.
\end{itemize}

Dokumentieren Sie kurz, warum Ihr Layout trotz Abständen übersichtlich bleibt.

\section*{Thema 4: CSS Positionierung}

\subsection*{Aufgabe 4.1 -- Positionierungsarten beschreiben}
Erklären Sie die Unterschiede zwischen:
\begin{itemize}
  \item \texttt{position: static}
  \item \texttt{position: relative}
  \item \texttt{position: absolute}
  \item \texttt{position: fixed}
\end{itemize}

Ergänzen Sie pro Positionierungsart ein typisches Einsatzbeispiel.

\subsection*{Aufgabe 4.2 -- Relative Positionierung anwenden}
Erstellen Sie einen Container mit einer Box.

Setzen Sie die Box auf \texttt{position: relative;} und verschieben Sie sie mit:
\begin{itemize}
  \item \texttt{top}
  \item \texttt{left}
\end{itemize}

Beschreiben Sie:
\begin{itemize}
  \item Von welchem Ursprungsort verschoben wird.
  \item Ob der ursprüngliche Platz im Dokumentfluss erhalten bleibt.
\end{itemize}

\subsection*{Aufgabe 4.3 -- Absolute Positionierung}
Erstellen Sie folgenden Aufbau:
\begin{itemize}
  \item Ein Container mit \texttt{position: relative;}
  \item Zwei Kind-Elemente mit \texttt{position: absolute;}
\end{itemize}

Positionieren Sie die Kinder an unterschiedliche Ecken.

Entfernen Sie danach testweise \texttt{position: relative;} beim Container und beschreiben Sie den Unterschied.

\subsection*{Aufgabe 4.4 -- Fixed und Scrollverhalten}
Erstellen Sie eine Seite mit genügend Inhalt zum Scrollen.

Fügen Sie ein fixes Element hinzu (z.\,B. Hinweisbox oder Top-Bar) mit \texttt{position: fixed;} und passenden Offset-Werten.

Analysieren Sie:
\begin{itemize}
  \item Was passiert beim Scrollen?
  \item Welchen Einfluss hat das feste Element auf die Nutzbarkeit der Seite?
\end{itemize}

\subsection*{Aufgabe 4.5 -- Display und Dokumentfluss}
Erstellen Sie drei Boxen mit unterschiedlichem Inhalt untereinander.

Testen Sie nacheinander bei allen Boxen gleichzeitig:
\begin{itemize}
  \item \texttt{display: block}
  \item \texttt{display: inline}
  \item \texttt{display: inline-block}
\end{itemize}

Passen Sie nun das CSS einzelner Boxen an:
\begin{itemize}
  \item Kombination mit \texttt{position: relative} und \texttt{position: absolute}
\end{itemize}

Beschreiben Sie für jeden Fall den Einfluss auf den Dokumentfluss.

\subsection*{Aufgabe 4.6 -- Transferaufgabe Positionierung}
Erstellen Sie eine Mini-Webseite mit:
\begin{itemize}
  \item fixer Kopfzeile
  \item Hauptbereich mit relativ positioniertem Container
  \item zwei absolut positionierten Markierungen innerhalb dieses Containers
  \item mindestens einem Element, das normal im Dokumentfluss bleibt
\end{itemize}

Ergänzen Sie eine kurze Reflexion (5--6 Sätze):
\begin{itemize}
  \item Wann ist absolute Positionierung sinnvoll?
  \item Wann sollte man sie vermeiden?
  \item Welche Rolle spielt \texttt{display} im Zusammenspiel mit \texttt{position}?
\end{itemize}

\section*{Abschlussaufgabe (optional)}

Erstellen Sie eine kompakte Lernseite, in der alle vier Themen sichtbar vorkommen:
\begin{itemize}
  \item semantische HTML-Struktur
  \item verschiedene CSS-Selektoren
  \item bewusst eingesetztes Box-Modell
  \item mindestens zwei Positionierungsarten
\end{itemize}

Ziel: Ein sauberes, gut lesbares Mini-Projekt, das die gesamte Repetition zusammenfasst.

\end{document}
