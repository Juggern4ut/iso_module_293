\documentclass[a4paper,12pt]{article}
\usepackage[utf8]{inputenc}
\usepackage[ngerman]{babel}
\usepackage{amsmath}
\usepackage{parskip}

\setlength{\parindent}{0pt}

\title{Modul 293 - Lösungsblatt 08}
\author{von Lukas Meier}
\date{Unterricht vom 10.02.2026}

\begin{document}

\maketitle

\section*{Lösungsblatt: Repetition HTML und CSS}

\section*{Thema 1: HTML Grundlagen}

\subsection*{Lösung 1.1 -- Begriffe sicher unterscheiden}
\begin{itemize}
  \item Ein \textbf{HTML-Tag} ist eine Markierung in spitzen Klammern, die dem Browser sagt, welche Rolle ein Inhalt hat, z.\,B. \texttt{<p>} für einen Absatz.
  \item Ein \textbf{Attribut} liefert zusätzliche Informationen zu einem Tag, z.\,B. \texttt{href} bei einem Link oder \texttt{alt} bei einem Bild.
  \item \textbf{Start-Tag}: öffnet das Element (z.\,B. \texttt{<h1>}), \textbf{End-Tag}: schliesst es (\texttt{</h1>}), \textbf{Inhalt}: steht dazwischen (z.\,B. \texttt{Titel}).
\end{itemize}

Beispiel:
\begin{verbatim}
<h1 class="titel">Willkommen</h1>
\end{verbatim}
Tag: \texttt{h1}, Attribut: \texttt{class="titel"}, Inhalt: \texttt{Willkommen}.

\subsection*{Lösung 1.2 -- Tags erkennen und einordnen}
Code:
\begin{verbatim}
<a href="https://example.com" target="_blank">Zur Website</a>
<img src="bild.jpg" alt="Profilbild">
<section class="intro">Willkommen</section>
\end{verbatim}

\begin{itemize}
  \item Tags: \texttt{<a> ... </a>}, \texttt{<img>}, \texttt{<section> ... </section>}.
  \item Attribute: \texttt{href}, \texttt{target}, \texttt{src}, \texttt{alt}, \texttt{class}.
  \item Selbstschliessendes Element in diesem Ausschnitt: \texttt{<img>}.
\end{itemize}

\subsection*{Lösung 1.3 -- Selbstschliessende Tags}
Eine mögliche Lösung (nur Body-Inhalt):
\begin{verbatim}
<h1>Mein Lerntagebuch</h1>
<p>Heute wiederhole ich HTML und CSS.<br />Ich übe Schritt für Schritt.</p>
<img src="lernen.jpg" alt="Schreibtisch mit Laptop und Notizen">
<hr />
\end{verbatim}

Selbstschliessende Tags hier: \texttt{<br />}, \texttt{<img />}, \texttt{<hr />}.

\subsection*{Lösung 1.4 -- Semantik verbessern}
Eine mögliche semantische Struktur:
\begin{verbatim}
<header>
  <h1>Meine Lernseite</h1>
</header>
<nav>
  <a href="#">Start</a>
  <a href="#">Kontakt</a>
</nav>
<main>
  <section>
    <p>Heute üben wir HTML.</p>
  </section>
</main>
\end{verbatim}

\subsection*{Lösung 1.5 -- Fehleranalyse}
Korrigierter Code:
\begin{verbatim}
<section>
  <h1 class="titel">Wochenplan</h1>
  <p id="text" class="wichtig">Montag: CSS wiederholen</p>
  <img src="plan.png" alt="Wochenplan als Bild" />
</section>
\end{verbatim}

Vier Fehler und Begründung:
\begin{itemize}
  \item \texttt{<h1>} wurde nicht geschlossen: Überschriften mit Inhalt brauchen ein End-Tag.
  \item \texttt{"wichtig"} war ein ungültiges freies Attributfragment: für Klassennamen braucht es \texttt{class="..."}.
  \item Bei \texttt{<img>} war \texttt{source} falsch: korrekt ist \texttt{src}.
  \item Der Selbtschliessende \texttt{img}-Tag wurde am Ende nicht mit einem \texttt{/} geschlossen.
\end{itemize}

\subsection*{Lösung 1.6 -- Transferaufgabe HTML}
Eine vollständige Mini-Seite:
\begin{verbatim}
<body>
  <header>
    <h1>Lektion 08 Repetition</h1>
  </header>

  <nav>
    <a href="thema1.html">Thema 1</a>
    <a href="thema2.html">Thema 2</a>
    <a href="kontakt.html">Kontakt</a>
  </nav>

  <main>
    <section id="thema1">
      <h2>HTML</h2>
      <p>Hier wiederholen wir Tags und Attribute.</p>
      <hr>
    </section>

    <section id="thema2">
      <h2>CSS</h2>
      <p>Hier wiederholen wir Selektoren und Layout.</p>
      <img src="css.png" alt="CSS Logo">
    </section>
  </main>

  <footer id="kontakt">
    <p>copyright 2026 Modul 293</p>
  </footer>
</body>
\end{verbatim}

\section*{Thema 2: CSS Selektoren}

\subsection*{Lösung 2.1 -- Grundprinzipien}
\begin{itemize}
  \item Tag-Selektor: direkt mit dem Tag-Namen, z.\,B. \texttt{p \{ color: blue; \}}.
  \item Klassen-Selektor: mit Punkt, z.\,B. \texttt{.info \{ color: green; \}}.
  \item ID-Selektor: mit Raute, z.\,B. \texttt{\#title \{ color: red; \}}.
\end{itemize}

Mini-Beispiel:
\begin{verbatim}
<h1 id="title">Titel</h1>
<p class="info">Text</p>

h1 { font-size: 32px; }
.info { color: green; }
#title { text-transform: uppercase; }
\end{verbatim}

\subsection*{Lösung 2.2 -- Selektoren zuordnen}
Gegebenes HTML:
\begin{verbatim}
<h1 id="title">Modul 293</h1>
<p class="info">Einführung</p>
<p class="info">Repetition</p>
<button class="info" id="start-btn">Start</button>
\end{verbatim}

Selektierte Elemente:
\begin{itemize}
  \item \texttt{p} selektiert die beiden Absatz-Elemente.
  \item \texttt{.info} selektiert beide \texttt{p} und den \texttt{button}.
  \item \texttt{\#title} selektiert nur das \texttt{h1}-Element.
  \item \texttt{button.info} selektiert nur den \texttt{button} mit Klasse \texttt{info}.
\end{itemize}

\subsection*{Lösung 2.3 -- Mehrere Elemente in einem Selektor}
Mögliche CSS-Lösung:
\begin{verbatim}
h1, h2, h3 {
  color: #1f4e79;
}

p, li {
  font-size: 18px;
}

.note, #warning {
  background-color: #fff3cd;
}
\end{verbatim}

\subsection*{Lösung 2.4 -- Kombinierte Selektoren}
Mögliche CSS-Regeln:
\begin{verbatim}
p.info {
  color: blue;
}

#menu {
  border: 2px solid #333;
}

.nav li {
  font-weight: bold;
}
\end{verbatim}

Warum \texttt{.info} zu ungenau ist:
\begin{itemize}
  \item \texttt{.info} trifft alle Elemente mit dieser Klasse (z.\,B. \texttt{p}, \texttt{div}, \texttt{button}).
  \item Mit \texttt{p.info} wird gezielt nur der Absatz mit dieser Klasse angesprochen.
\end{itemize}

\subsection*{Lösung 2.5 -- Fehler finden und korrigieren}
Korrigierte Variante:
\begin{verbatim}
#title {
  color: red;
}

.intro {
  font-size: 20px;
}

p.hinweis, #wichtig {
  background: yellow;
}
\end{verbatim}

Kurzbegründung pro Block:
\begin{itemize}
  \item \texttt{title} würde das HTML-\texttt{<title>}-Element meinen; gemeint war hier typischerweise die ID \texttt{title}.
  \item \texttt{\#.intro} ist ungültig, da \texttt{\#} und \texttt{.} nicht gleichzeitig für denselben einfachen Selektor stehen.
  \item \texttt{p, .hinweis \#wichtig} Logikfehler! Id's sollen immer nur einmal im HTML vorkommen. Ein Selektor für alle Elemente mit der ID wichtig in allen Klassen hinweis macht also keinen Sinn.
\end{itemize}

\subsection*{Lösung 2.6 -- Transferaufgabe Selektoren}
Möglicher HTML-Ausschnitt:
\begin{verbatim}
<h1 id="page-title">Repetition</h1>
<h2 class="subtitle">Selektoren</h2>

<p class="info">Absatz 1</p>
<p>Absatz 2</p>
<p class="info important">Absatz 3</p>

<ul class="nav">
  <li>Home</li>
  <li>HTML</li>
  <li>CSS</li>
  <li>Kontakt</li>
</ul>

<button id="start">Start</button>
\end{verbatim}

Passendes CSS:
\begin{verbatim}
/* Tag-Selektor */
p {
  line-height: 1.5;
}

/* Klassen-Selektoren */
.info {
  color: #0a4;
}

.subtitle {
  letter-spacing: 0.5px;
}

/* ID-Selektor */
#start {
  border: 1px solid #333;
}

/* Gruppierte Selektoren */
h1, h2 {
  font-family: Arial, sans-serif;
}

ul, p {
  margin-bottom: 12px;
}

/* Kombinierter Selektor */
p.info {
  font-weight: 600;
}
\end{verbatim}

\section*{Thema 3: CSS Box-Modell}

\subsection*{Lösung 3.1 -- Theorie verstehen}
Reihenfolge von innen nach aussen:
\begin{itemize}
  \item \textbf{Content} (Inhalt)
  \item \textbf{Padding} (Innenabstand)
  \item \textbf{Border} (Rahmen)
  \item \textbf{Margin} (Aussenabstand)
\end{itemize}

Erklärungen:
\begin{itemize}
  \item \texttt{padding} ist der Abstand zwischen Inhalt und Rahmen.
  \item \texttt{margin} ist der Abstand eines Elements nach aussen zu anderen Elementen.
  \item Verwendete Properties: \texttt{padding}, \texttt{padding-top/right/bottom/left}, \texttt{margin}, \texttt{margin-top/right/bottom/left}.
\end{itemize}

\subsection*{Lösung 3.2 -- Box aufbauen}
CSS:
\begin{verbatim}
.card {
  width: 280px;
  padding: 16px;
  border: 2px solid black;
  margin: 20px;
}
\end{verbatim}

Skizze (textuell):
\begin{itemize}
  \item Mitte: Inhaltstext
  \item Um den Inhalt: 16px Padding
  \item Danach: 2px Border
  \item Aussen: 20px Margin zur Umgebung
\end{itemize}

\subsection*{Lösung 3.3 -- Margin vs. Padding gezielt testen}
Beobachtung:
\begin{itemize}
  \item Wenn nur \texttt{padding} grösser wird, wächst der innere Abstand zwischen Inhalt und Rahmen; die Box wirkt innen luftiger.
  \item Wenn nur \texttt{margin} grösser wird, bleibt das Innere gleich, aber der Abstand zu anderen Elementen wird grösser.
  \item \texttt{padding} beeinflusst den Innenraum der Box, \texttt{margin} den Aussenraum im Layout.
\end{itemize}

\subsection*{Lösung 3.4 -- Block und Inline vergleichen}
\begin{itemize}
  \item \texttt{div} ist standardmässig \texttt{display: block}.
  \item \texttt{span} ist standardmässig \texttt{display: inline}.
  \item Bei \texttt{span} wirken \texttt{width}, \texttt{height} und vertikale \texttt{margin} in der Regel nicht wie bei Block-Elementen.
  \item Mit \texttt{display: inline-block} akzeptiert \texttt{span} Breite/Höhe und bleibt trotzdem im Textfluss.
\end{itemize}

\subsection*{Lösung 3.5 -- Rechnen mit dem Box-Modell}
Gegeben: \texttt{width: 300px}, links/rechts \texttt{padding: 20px}, links/rechts \texttt{border: 3px}.

\textbf{content-box}: 
\begin{itemize}
  \item Gesamtbreite = \(300 + 20 + 20 + 3 + 3 = 346\)px
\end{itemize}

\textbf{border-box}: 
\begin{itemize}
  \item Gesamtbreite = \(300\)px (Padding und Border sind in den 300px enthalten)
\end{itemize}

Unterschied: Bei \texttt{content-box} kommt Padding/Border zur Breite dazu, bei \texttt{border-box} nicht.

\subsection*{Lösung 3.6 -- Transferaufgabe Box-Modell}
Mögliche Lösung:
\begin{verbatim}
<section class="cards">
  <article class="card card-a">
    <h3>Karte A</h3>
    <p>Mehr Padding, kleine Margin.</p>
  </article>

  <article class="card card-b">
    <h3>Karte B</h3>
    <p>Kleine Padding, grosse Margin.</p>
  </article>

  <article class="card card-c">
    <h3>Karte C</h3>
    <p>Inline-block Karte.</p>
  </article>
</section>
\end{verbatim}

\begin{verbatim}
.card {
  box-sizing: border-box;
  width: 260px;
  border: 2px solid #222;
}

.card-a {
  padding: 24px;
  margin: 8px;
}

.card-b {
  padding: 10px;
  margin: 24px;
}

.card-c {
  padding: 16px;
  margin: 12px;
  display: inline-block;
}
\end{verbatim}

Kurze Dokumentation:
\begin{itemize}
  \item Die Karten bleiben übersichtlich, weil \texttt{box-sizing: border-box} die Breite stabil hält.
  \item Unterschiedliche Margin/Padding-Werte verändern gezielt den Innen- und Aussenabstand.
\end{itemize}

\section*{Thema 4: CSS Positionierung}

\subsection*{Lösung 4.1 -- Positionierungsarten beschreiben}
\begin{itemize}
  \item \texttt{static}: Standard, Element bleibt normal im Dokumentfluss, \texttt{top/right/bottom/left} wirken nicht.
  \item \texttt{relative}: Element bleibt im Fluss, kann aber visuell relativ zu seiner Ausgangsposition verschoben werden.
  \item \texttt{absolute}: Element wird aus dem normalen Fluss genommen und relativ zum nächsten positionierten Vorfahren platziert.
  \item \texttt{fixed}: Element wird relativ zum Viewport fixiert und bleibt beim Scrollen sichtbar.
\end{itemize}

Einsatzbeispiele:
\begin{itemize}
  \item \texttt{static}: normale Text- und Inhaltsblöcke
  \item \texttt{relative}: kleine Korrekturpositionen bei Badges/Labels
  \item \texttt{absolute}: Icons oder Marker in einer Karte
  \item \texttt{fixed}: Sticky-Topbar oder dauerhaft sichtbarer Hilfe-Button
\end{itemize}

\subsection*{Lösung 4.2 -- Relative Positionierung anwenden}
Beispiel:
\begin{verbatim}
.container {
  width: 400px;
  border: 1px solid #888;
}

.box {
  position: relative;
  top: 10px;
  left: 20px;
}
\end{verbatim}

Beschreibung:
\begin{itemize}
  \item Verschoben wird vom ursprünglichen Platz der Box aus.
  \item Der ursprüngliche Platz im Dokumentfluss bleibt reserviert.
\end{itemize}

\subsection*{Lösung 4.3 -- Absolute Positionierung}
Beispiel:
\begin{verbatim}
.container {
  position: relative;
  width: 500px;
  height: 220px;
  border: 2px solid #333;
}

.marker-a {
  position: absolute;
  top: 10px;
  left: 10px;
}

.marker-b {
  position: absolute;
  bottom: 10px;
  right: 10px;
}
\end{verbatim}

Unterschied beim Entfernen von \texttt{position: relative} am Container:
\begin{itemize}
  \item Die Marker orientieren sich nicht mehr am Container, sondern am nächsten anderen positionierten Vorfahren (oder am Viewport).
\end{itemize}

\subsection*{Lösung 4.4 -- Fixed und Scrollverhalten}
Beispiel:
\begin{verbatim}
.notice {
  position: fixed;
  top: 12px;
  right: 12px;
}
\end{verbatim}

Analyse:
\begin{itemize}
  \item Das Element bleibt beim Scrollen sichtbar an derselben Stelle im Fenster.
  \item Es verbessert die Sichtbarkeit wichtiger Infos, darf aber Inhalte nicht verdecken.
\end{itemize}

\subsection*{Lösung 4.5 -- Display und Dokumentfluss}
Beobachtung:
\begin{itemize}
  \item \texttt{display: block}: jedes Element startet in neuer Zeile, nimmt standardmässig volle Breite ein.
  \item \texttt{display: inline}: Elemente liegen in einer Zeile, Breite/Höhe nur eingeschränkt steuerbar.
  \item \texttt{display: inline-block}: Elemente bleiben in einer Zeile, Breite/Höhe sind aber steuerbar.
  \item \texttt{position: relative}: Element bleibt im Fluss, nur visuell verschoben.
  \item \texttt{position: absolute}: Element verlässt den normalen Fluss und überlagert ggf. andere Elemente.
\end{itemize}

\subsection*{Lösung 4.6 -- Transferaufgabe Positionierung}
Mögliche Mini-Webseite:
\begin{verbatim}
<header class="topbar">Modul 293</header>

<main class="page-content">
  <section class="board">
    <span class="pin pin-a">A</span>
    <span class="pin pin-b">B</span>
    <article class="text-block">
      Dieses Element bleibt normal im Dokumentfluss.
    </article>
  </section>
</main>
\end{verbatim}

\begin{verbatim}
.topbar {
  position: fixed;
  top: 0;
  left: 0;
  right: 0;
  height: 52px;
}

.page-content {
  margin-top: 70px;
}

.board {
  position: relative;
  min-height: 220px;
  border: 2px solid #444;
}

.pin {
  position: absolute;
  width: 28px;
  height: 28px;
  display: inline-block;
  text-align: center;
}

.pin-a { top: 12px; left: 12px; }
.pin-b { bottom: 12px; right: 12px; }
\end{verbatim}

Reflexion (Beispiel):
\begin{itemize}
  \item Absolute Positionierung ist sinnvoll für dekorative Marker, Overlays und exakt platzierte UI-Elemente.
  \item Sie sollte vermieden werden, wenn Inhalte dynamisch wachsen oder stark responsiv sein müssen.
  \item Zu viele absolute Elemente führen schnell zu überlappenden Inhalten.
  \item \texttt{display} steuert weiterhin, wie ein Element grundsätzlich im Layout erscheint (Block, Inline, Inline-Block).
  \item \texttt{position} steuert dagegen, wie es im oder ausserhalb des Dokumentflusses platziert wird.
  \item Erst das Zusammenspiel beider Eigenschaften ergibt ein stabiles Layout.
\end{itemize}

\section*{Lösung Abschlussaufgabe (optional)}
Eine mögliche kompakte Lernseite erfüllt alle Kriterien durch:
\begin{itemize}
  \item Semantik: \texttt{header}, \texttt{nav}, \texttt{main}, \texttt{section}, \texttt{footer}
  \item Selektoren: Kombination aus \texttt{h1}, \texttt{.info}, \texttt{\#menu}, \texttt{p.info}, \texttt{h1, h2}
  \item Box-Modell: Karten mit \texttt{padding}, \texttt{margin}, \texttt{border}, global \texttt{box-sizing: border-box}
  \item Positionierung: fixe Kopfzeile plus absolut positionierte Marker in einem relativ positionierten Container
\end{itemize}

\end{document}
