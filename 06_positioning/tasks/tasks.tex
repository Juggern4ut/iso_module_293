\documentclass[a4paper,12pt]{article}
\usepackage[utf8]{inputenc}
\usepackage[ngerman]{babel}
\usepackage{amsmath}
\usepackage{parskip}

\setlength{\parindent}{0pt}

\title{Modul 293 - Aufgabenblatt 06}
\author{von Lukas Meier}
\date{Unterricht vom 27.01.2026}

\begin{document}
\section*{Aufgabenblatt: Positionierung von Elementen mit CSS}

\subsection*{Teil 1: Grundlagen verstehen}

Beantworte die folgenden Fragen schriftlich:

\begin{itemize}
  \item Was bedeutet \texttt{position: static;}?
  \item Welche CSS-Eigenschaften (\texttt{top}, \texttt{left}, \texttt{right}, \texttt{bottom}) haben bei \texttt{static} keine Wirkung?
\end{itemize}

\subsection*{Teil 2: Relative Positionierung}

Gegeben ist folgendes CSS:

\begin{verbatim}
.box {
  position: relative;
  top: 20px;
  left: 30px;
}
\end{verbatim}

\begin{itemize}
  \item Beschreibe, von welchem Punkt aus das Element verschoben wird.
  \item Bleibt der ursprüngliche Platz des Elements im Dokument erhalten?
\end{itemize}

Erstelle ein eigenes Beispiel mit:
\begin{itemize}
  \item einem Container
  \item einem relativ positionierten Element
\end{itemize}

\subsection*{Teil 3: Absolute Positionierung}

Erkläre den Unterschied zwischen \texttt{position: relative;} und \texttt{position: absolute;}.

Gegeben ist folgender HTML-Code:

\begin{verbatim}
<div class="container">
  <div class="box"></div>
</div>
\end{verbatim}

Und folgendes CSS:

\begin{verbatim}
.container {
  position: relative;
}

.box {
  position: absolute;
  top: 10px;
  right: 10px;
}
\end{verbatim}

\begin{itemize}
  \item Relativ zu welchem Element wird die Box positioniert?
  \item Was passiert, wenn \texttt{position: relative;} beim Container entfernt wird?
\end{itemize}

\subsection*{Teil 4: Fixe Positionierung}

Beschreibe in eigenen Worten, wie sich \texttt{position: fixed;} verhält.

\begin{itemize}
  \item Was passiert beim Scrollen der Seite?
  \item Nenne zwei typische Einsatzgebiete für fixe Positionierung.
\end{itemize}

Erstelle ein Beispiel für:
\begin{itemize}
  \item eine fixe Navigationsleiste
  \item ein fixes Informationsfeld
\end{itemize}

\subsection*{Teil 5: Z-Index verstehen}

Gegeben sind zwei überlappende Boxen:

\begin{verbatim}
.box1 {
  position: absolute;
  z-index: 1;
}

.box2 {
  position: absolute;
  z-index: 2;
}
\end{verbatim}

\begin{itemize}
  \item Welche Box liegt im Vordergrund?
  \item Warum funktioniert \texttt{z-index} nur bei positionierten Elementen?
\end{itemize}

Erstelle ein eigenes Beispiel mit mindestens drei Elementen und unterschiedlichen \texttt{z-index}-Werten.

\subsection*{Teil 6: Fehler finden und korrigieren}

Der folgende CSS-Code enthält konzeptionelle Fehler:

\begin{verbatim}
.box {
  z-index: 10;
  top: 20px;
}
\end{verbatim}

\begin{itemize}
  \item Erkläre, warum der Code nicht wie erwartet funktioniert.
  \item Korrigiere den Code so, dass er sinnvoll eingesetzt werden kann.
\end{itemize}

\subsection*{Teil 7: Praxisaufgabe}

Erstelle eine HTML-Seite mit folgenden Anforderungen:

\begin{itemize}
  \item Ein Container mit relativer Positionierung
  \item Mindestens zwei absolut positionierte Elemente darin
  \item Ein fixes Element, das beim Scrollen sichtbar bleibt
  \item Überlappende Elemente mit sinnvoll eingesetztem \texttt{z-index}
\end{itemize}

Achte auf saubere Struktur und lesbaren Code.

\subsection*{Teil 8: Reflexion}

Beantworte die folgenden Fragen:

\begin{itemize}
  \item Welche Positionierungsart war am schwierigsten zu verstehen?
  \item In welchen Situationen würdest du absolute Positionierung vermeiden?
  \item Warum ist ein gutes Verständnis von Positionierung wichtig für Layouts?
\end{itemize}

\end{document}
