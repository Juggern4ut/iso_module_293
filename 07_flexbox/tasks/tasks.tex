\documentclass[a4paper,12pt]{article}
\usepackage[utf8]{inputenc}
\usepackage[ngerman]{babel}
\usepackage{amsmath}
\usepackage{parskip}

\setlength{\parindent}{0pt}

\title{Modul 293 - Aufgabenblatt 07}
\author{von Lukas Meier}
\date{Unterricht vom 03.02.2026}

\begin{document}
\section*{Aufgabenblatt: Flexbox in CSS}

\subsection*{Teil 1: Grundlagen verstehen}

Beantworte die folgenden Fragen schriftlich:

\begin{itemize}
  \item Was ist der Unterschied zwischen einem normalen Block-Layout und einem Flex-Layout?
  \item Welche Elemente werden zu Flex-Items?
  \item Welche Eigenschaft aktiviert Flexbox auf einem Container?
\end{itemize}

\subsection*{Teil 2: Achsen verstehen}

Gegeben ist folgendes CSS:

\begin{verbatim}
.container {
  display: flex;
  flex-direction: row;
}
\end{verbatim}

\begin{itemize}
  \item In welche Richtung verläuft die Hauptachse?
  \item Wo liegt in diesem Fall die Querachse?
\end{itemize}

Ändere anschliessend \texttt{flex-direction} auf \texttt{column} und beschreibe die Unterschiede.

\subsection*{Teil 3: Ausrichtung im Container}

Erstelle ein eigenes Beispiel mit mindestens drei Boxen und einem Flex-Container.

Teste nacheinander:
\begin{itemize}
  \item \texttt{justify-content: center;}
  \item \texttt{justify-content: space-between;}
  \item \texttt{align-items: center;}
\end{itemize}

Dokumentiere jeweils kurz, was sich sichtbar verändert.

\subsection*{Teil 4: Umbruch und Abstände}

Gegeben ist folgendes CSS:

\begin{verbatim}
.container {
  display: flex;
  flex-wrap: wrap;
  gap: 12px;
}
\end{verbatim}

\begin{itemize}
  \item Was passiert, wenn der Platz in einer Zeile nicht mehr ausreicht?
  \item Welchen Vorteil hat \texttt{gap} gegenüber Margins auf den einzelnen Items?
\end{itemize}

Erstelle ein Beispiel mit mindestens sechs Elementen, sodass ein Zeilenumbruch sichtbar wird.

\subsection*{Teil 5: Eigenschaften von Flex-Items}

Untersuche folgende Werte:

\begin{verbatim}
.item-a { flex: 1; }
.item-b { flex: 2; }
.item-c { flex: 1; }
\end{verbatim}

\begin{itemize}
  \item Welches Element bekommt am meisten Platz?
  \item Wie verändert sich das Layout, wenn alle Elemente \texttt{flex: 1;} erhalten?
\end{itemize}

\subsection*{Teil 6: Fehler finden und korrigieren}

Der folgende Code enthält konzeptionelle Fehler:

\begin{verbatim}
.item {
  justify-content: center;
  align-items: center;
}
\end{verbatim}

\begin{itemize}
  \item Warum funktionieren diese Eigenschaften hier nicht wie erwartet?
  \item Korrigiere den Code so, dass die Ausrichtung korrekt funktioniert.
\end{itemize}

\subsection*{Teil 7: Praxisaufgabe}

Erstelle eine kleine Startseite mit:

\begin{itemize}
  \item einer Navigationsleiste mit Logo und drei Links
  \item einem Kartenbereich mit mindestens vier Karten
  \item einem Footer
\end{itemize}

Verwende Flexbox für alle drei Bereiche und achte auf:
\begin{itemize}
  \item saubere Struktur
  \item sinnvolle Klassennamen
  \item lesbaren und eingerückten Code
\end{itemize}

\subsection*{Teil 8: Reflexion}

Beantworte die folgenden Fragen:

\begin{itemize}
  \item Welche Flexbox-Eigenschaft war für dich am hilfreichsten?
  \item Bei welcher Aufgabe hattest du Schwierigkeiten und warum?
  \item In welcher realen Webseite könntest du Flexbox direkt einsetzen?
\end{itemize}

\end{document}
