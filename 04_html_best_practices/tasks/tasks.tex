\documentclass[a4paper,12pt]{article}
\usepackage[utf8]{inputenc}
\usepackage[ngerman]{babel}
\usepackage{amsmath}
\usepackage{parskip}

\setlength{\parindent}{0pt} % Keine Einrückungen

\title{Modul 293 - Aufgabenblatt 03}

\author{von Lukas Meier}

\date{Unterricht vom 09.12.2025}

\begin{document}

\maketitle

\section*{Aufgabe: HTML Best Practices}

Ziel dieser Aufgabe ist es, grundlegende Best Practices im Umgang mit HTML zu erkennen und anzuwenden.

\subsection*{Teil 1: Theoriefragen}

Beantworte die folgenden Fragen in ganzen Saetzen:

\begin{enumerate}
  \item Warum ist es wichtig, semantische HTML-Tags zu verwenden?
  \item Weshalb sollte man HTML und CSS voneinander trennen?
  \item Nenne zwei Vorteile von sauber eingeruecktem HTML-Code.
  \item Warum sollte jede HTML-Seite ein \texttt{<!DOCTYPE html>} enthalten?
\end{enumerate}

\subsection*{Teil 2: HTML analysieren und korrigieren}

Du hast eine HTML-Datei erhalten, welche absichtlich nicht den Best Practices entspricht.

\textbf{Aufgabe:}
\begin{itemize}
  \item Analysiere den HTML-Code.
  \item Markiere mindestens \textbf{fuenf Probleme} im Code.
  \item Korrigiere den Code, sodass er den HTML Best Practices entspricht.
\end{itemize}

Achte dabei insbesondere auf:
\begin{itemize}
  \item Semantische HTML-Tags
  \item Ueberschriftenstruktur
  \item Einrueckung und Lesbarkeit
  \item Aussagekraeftige Inhalte
  \item Trennung von Struktur und Design
\end{itemize}

\subsection*{Teil 3: Semantik verbessern}

Erweitere den von dir korrigierten HTML-Code aus Teil 2 weiter.

\begin{itemize}
  \item Ersetze allgemeine \texttt{div}-Tags, wo sinnvoll, durch semantische HTML-Tags.
  \item Verwende mindestens drei unterschiedliche semantische Tags (z.B. \texttt{header}, \texttt{main}, \texttt{section}, \texttt{footer}).
\end{itemize}

Begruende kurz schriftlich, weshalb du diese Tags gewaehlt hast.

\subsection*{Teil 4: Ueberschriftenstruktur}

Ueberpruefe die Ueberschriftenstruktur deiner HTML-Seite.

\begin{itemize}
  \item Stelle sicher, dass genau ein \texttt{h1}-Tag vorhanden ist.
  \item Ordne weitere Ueberschriften logisch darunter ein.
\end{itemize}

Warum ist eine korrekte Ueberschriftenhierarchie wichtig fuer:
\begin{itemize}
  \item Suchmaschinen?
  \item Screenreader?
\end{itemize}

\subsection*{Teil 5: Barrierefreiheit (Grundlagen)}

Verbessere die Zugaenglichkeit deiner HTML-Seite.

\begin{itemize}
  \item Fuege allen Bildern ein sinnvolles \texttt{alt}-Attribut hinzu.
  \item Verwende aussagekraeftige Linktexte anstelle von generischen Begriffen.
\end{itemize}

Erklaere in 2--3 Saetzen, weshalb Barrierefreiheit im Web wichtig ist.

\subsection*{Teil 6: Code-Qualitaet}

Formatiere deinen HTML-Code so, dass er gut lesbar ist.

\begin{itemize}
  \item Verwende konsistente Einrueckungen.
  \item Entferne unnoetige Leerzeilen.
  \item Achte auf eine saubere Struktur.
\end{itemize}

Vergleiche deinen formatierten Code mit der urspruenglichen Version.  
Welche Vorteile ergeben sich durch gut strukturierten HTML-Code?

\subsection*{Teil 7: Reflexion}

Beantworte die folgenden Fragen schriftlich:

\begin{itemize}
  \item Welche Best Practice war fuer dich neu?
  \item Welche Aenderung hatte den groessten Einfluss auf die Qualitaet des Codes?
  \item Worauf moechtest du bei zukuenftigem HTML-Code besonders achten?
\end{itemize}

\end{document}
