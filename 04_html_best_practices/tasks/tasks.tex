\documentclass[a4paper,12pt]{article}
\usepackage[utf8]{inputenc}
\usepackage[ngerman]{babel}
\usepackage{amsmath}
\usepackage{parskip}

\setlength{\parindent}{0pt} % Keine Einrückungen

\title{Modul 293 - Aufgabenblatt 04}

\author{von Lukas Meier}

\date{Unterricht vom 13.01.2026}

\begin{document}

\maketitle

\section*{Aufgabe: HTML Best Practices}

Ziel dieser Aufgabe ist es, grundlegende Best Practices im Umgang mit HTML zu erkennen und anzuwenden.

\subsection*{Teil 1: Theoriefragen}

Beantworte die folgenden Fragen in ganzen Saetzen:

\begin{enumerate}
  \item Warum ist es wichtig, semantische HTML-Tags zu verwenden?
  \item Weshalb sollte man HTML und CSS voneinander trennen?
  \item Nenne zwei Vorteile von sauber eingerücktem HTML-Code.
  \item Warum sollte jede HTML-Seite ein \texttt{<!DOCTYPE html>} enthalten?
\end{enumerate}

\subsection*{Teil 2: HTML analysieren und korrigieren}

Sie haben eine HTML-Datei erhalten, welche absichtlich nicht den Best Practices entspricht.

\textbf{Aufgabe:}
\begin{itemize}
  \item Analysiere den HTML-Code.
  \item Markiere mindestens \textbf{fünf Probleme} im Code.
  \item Korrigiere den Code, sodass er den HTML Best Practices entspricht.
\end{itemize}

Achten Sie dabei besonders auf:
\begin{itemize}
  \item Semantische HTML-Tags
  \item Überschriftenstruktur
  \item Einrückung und Lesbarkeit
  \item Aussagekräftige Inhalte
  \item Trennung von Struktur und Design
\end{itemize}

\subsection*{Teil 3: Semantik verbessern}

Erweiteren Sie den korrigierten HTML-Code aus Teil 2 weiter.

\begin{itemize}
  \item Ersetzen Sie allgemeine \texttt{div}-Tags, wo sinnvoll, durch semantische HTML-Tags.
  \item Verwenden Sie mindestens drei unterschiedliche semantische Tags (z.B. \texttt{header}, \texttt{main}, \texttt{section}, \texttt{footer}).
\end{itemize}

Begründen Sie kurz schriftlich, weshalb Sie diese Tags gewählt haben.

\subsection*{Teil 4: Überschriftenstruktur}

Überpruefen Sie die Überschriftenstruktur Ihrer HTML-Seite.

\begin{itemize}
  \item Stellen Sie sicher, dass genau ein \texttt{h1}-Tag vorhanden ist.
  \item Ordnen Sie weitere Überschriften logisch darunter ein.
\end{itemize}

Warum ist eine korrekte Überschriftenhierarchie wichtig für:
\begin{itemize}
  \item Suchmaschinen?
  \item Screenreader?
\end{itemize}

\subsection*{Teil 5: Barrierefreiheit (Grundlagen)}

Verbessern Sie die Zugänglichkeit Ihrer HTML-Seite.

\begin{itemize}
  \item Fügen Sie allen Bildern ein sinnvolles \texttt{alt}-Attribut hinzu.
  \item Verwenden Sie aussagekräftige Linktexte anstelle von generischen Begriffen.
\end{itemize}

Erklären Sie in 2--3 Sätzen, weshalb Barrierefreiheit im Web wichtig ist.

\subsection*{Teil 6: Code-Qualität}

Formatieren Sie Ihren HTML-Code so, dass er gut lesbar ist.

\begin{itemize}
  \item Verwenden Sie konsistente Einrückungen.
  \item Entfernen Sie unnötige Leerzeilen.
  \item Achten Sie auf eine saubere Struktur.
\end{itemize}

Vergleichen Sie Ihren formatierten Code mit der ursprünglichen Version.  
Welche Vorteile ergeben sich durch gut strukturierten HTML-Code?

\subsection*{Teil 7: Reflexion}

Welche Änderung hatte für Sie den groessten Einfluss auf die Qualität des Codes?

\end{document}
