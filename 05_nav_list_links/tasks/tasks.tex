\documentclass[a4paper,12pt]{article}
\usepackage[utf8]{inputenc}
\usepackage[ngerman]{babel}
\usepackage{amsmath}
\usepackage{parskip}

\setlength{\parindent}{0pt}

\title{Modul 293 - Aufgabenblatt 04}
\author{von Lukas Meier}
\date{Unterricht vom \today}

\begin{document}

\maketitle

\section{Aufgaben}

In dieser Lektion haben Sie gelernt, wie Listen, Navigationen und Links in HTML korrekt umgesetzt werden.  
Ziel dieses Aufgabenblatts ist es, die Theorie praktisch anzuwenden und Best Practices kennenzulernen.

Bearbeiten Sie die folgenden Aufgaben der Reihe nach.

\subsection{Vorbereitung}

Öffnen Sie Ihr bestehendes Projekt mit der Datei \texttt{index.html}.  
Falls Sie noch keine funktionierende HTML-Datei haben, erstellen Sie eine neue Datei mit einer gültigen HTML-Grundstruktur.

\subsection{Ungeordnete Liste erstellen}

Erstellen Sie im \texttt{body} Ihrer Webseite eine ungeordnete Liste (\texttt{ul}) mit mindestens vier Einträgen.  
Die Liste soll eine Aufzählung Ihrer Lieblings-Webseiten oder -Themen darstellen.

Überlegen Sie sich:
\begin{itemize}
  \item Ist die Reihenfolge der Elemente relevant?
  \item Weshalb eignet sich hier eine ungeordnete Liste?
\end{itemize}

\subsection{Geordnete Liste erstellen}

Erstellen Sie unterhalb der ungeordneten Liste eine geordnete Liste (\texttt{ol}) mit mindestens drei Schritten.

Beispielhafte Themen:
\begin{itemize}
  \item Vorgehen beim Erstellen einer Webseite
  \item Schritte beim Programmieren
\end{itemize}

Vergleichen Sie visuell die Unterschiede zwischen \texttt{ul} und \texttt{ol} im Browser.

\subsection{Verschachtelte Liste}

Erweitern Sie eine bestehende Liste, indem Sie eine verschachtelte Liste hinzufügen.

Beispiel:
\begin{itemize}
  \item Hauptpunkt
    \begin{itemize}
      \item Unterpunkt
      \item Unterpunkt
    \end{itemize}
\end{itemize}

Achten Sie auf eine saubere Struktur und korrekte Einrückung im Code.

\subsection{Navigation erstellen}

Erstellen Sie eine Navigation für Ihre Webseite.

\begin{itemize}
  \item Verwenden Sie das semantische Element \texttt{nav}
  \item Innerhalb der Navigation verwenden Sie eine \texttt{ul}
  \item Jeder Menüpunkt soll ein \texttt{a}-Element sein
\end{itemize}

Die Navigation soll mindestens folgende Punkte enthalten:
\begin{itemize}
  \item Home
  \item Über uns
  \item Kontakt
\end{itemize}

\subsection{Interne Links}

Legen Sie für jeden Navigationspunkt eine eigene HTML-Datei an:
\begin{itemize}
  \item \texttt{index.html}
  \item \texttt{about.html}
  \item \texttt{contact.html}
\end{itemize}

Stellen Sie sicher, dass Sie mithilfe der Navigation zwischen allen Seiten wechseln können.

\subsection{Externer Link}

Fügen Sie Ihrer Navigation oder dem Seiteninhalt einen externen Link hinzu.

Vorgaben:
\begin{itemize}
  \item Der Link soll auf eine externe Webseite zeigen (z.B. https://google.com)
  \item Verwenden Sie das Attribut \texttt{target="\_blank"}
\end{itemize}

Beobachten Sie das Verhalten im Browser und erklären Sie den Unterschied zu internen Links.

\subsection{Best Practices für Linktexte}

Überprüfen Sie alle Linktexte Ihrer Webseite.

Beantworten Sie folgende Fragen:
\begin{itemize}
  \item Sind die Linktexte beschreibend?
  \item Würden die Links auch ohne Kontext Sinn ergeben?
  \item Gibt es generische Begriffe wie „hier klicken“?
\end{itemize}

Verbessern Sie die Linktexte falls nötig.

\subsection{Analyseaufgabe}

Öffnen Sie eine beliebige professionelle Webseite Ihrer Wahl.

Analysieren Sie:
\begin{itemize}
  \item Wie ist die Navigation aufgebaut?
  \item Werden Listen verwendet?
  \item Sind die Linktexte klar und verständlich?
\end{itemize}

Notieren Sie mindestens zwei Beobachtungen.

\subsection{Bonus: Erweiterung}

Erweitern Sie Ihre Navigation um einen weiteren Menüpunkt Ihrer Wahl.

Optional:
\begin{itemize}
  \item Fügen Sie eine verschachtelte Navigation hinzu (Untermenü)
  \item Recherchieren Sie, wie solche Menüs mit CSS gestaltet werden können
\end{itemize}

\end{document}
